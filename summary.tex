\documentclass[twocolumn]{article}

\usepackage[margin=1.2cm]{geometry}
\usepackage{amssymb}
\usepackage{amsmath}
\usepackage{todonotes}
\usepackage{xcolor}
\graphicspath{ {./images/} }
\usepackage{hyperref}
\hypersetup{
    colorlinks,
    citecolor=black,
    filecolor=black,
    linkcolor=black,
    urlcolor=black
}
\renewcommand{\labelenumi}{(\roman{enumi})}

\title{Zusammenfassung: Logik für die Informatik }
\author{Rico Klimpel}
\date{\today}

\begin{document}
    \maketitle   
    \tableofcontents
    \newpage

    \setcounter{page}{1}

    \section*{Informationen}
    Zusammenfassung der Vorlesung Logik für die Informatik an der CAU Kiel aus dem Wintersemester 2019/2020, gehalten von Prof. Dr. Thomas Wilke. Ein Versuch die wichtigsten Aussagen ohne enorm lange Formalitäten drum herum knapp zu Papier zu bringen. Kein Anspruch auf Vollständigkeit. Geschrieben in \LaTeX.

    \part{Aussagenlogik}
    Hier kommt alles zur Aussagenlogik rein.\\
    Ja\\
    Stimmt\\
    Schon ganz viel hier!

    \newpage
    \part{Prädikatenlogik}

    \section{S-Signatur}

    Eine Signatur $\mathcal S$ besteht aus eine Menge $S$ von Symbolen und einer Funktion $\Sigma \colon S \to \mathbf N \cup \mathbf N \times \{1\}$.\\
    The Elemente von $S$ werden Symbole genannt und wie folgt eingeteilt:
    \begin{itemize}
        \item Ein Symbol $f$ mit $\Sigma(f) = \langle n, 1\rangle$ für $n > 0$ ist eine Funktionssymbol.\\
        Menge dieser Symbole: $\mathcal F_\Sigma$ oder einfach $\mathcal F$.
        \item Ein Symbol $R$ mit $\Sigma(R) = n$ für $n > 0$ ist ein Relationssymbol.\\
        Menge dieser Symbole: $\mathcal R_\Sigma$ oder $\mathcal R$.
        \item Ein Symbol $c$ mit $\Sigma(c) = \langle 0,1\rangle$ ist ein Symbol für eine Konstante.\\
        Menge dieser Symbole: $\mathcal C_\Sigma$ oder $\mathcal C$.
        \item Symbol $b$ mit $\Sigma(b) = 0$ ist ein Symbol für einen boolschen Wert. \\
        Menge dieser Symbole: $\mathcal B_\Sigma$ or simply $\mathcal B$.
    \end{itemize}
    Im allgemeinen werden Signaturen mit $\mathcal B \neq \emptyset$ ignoriert (Signaturen ohne boolsche Werte).\\\\
    Beispiele:
    \begin{align*}
        S &= \{\text{zero}, \text{one}, \text{add}, \text{mult}\}\\
        \Sigma &= \{\text{zero} \mapsto \langle 0,1\rangle, \text{one} \mapsto \langle 0,1\rangle, \text{add} \mapsto \langle 2,1\rangle, \text{mult} \mapsto \langle 2,1\rangle\}
    \end{align*}
   Vereinfacht aufgeschrieben:
    $$\mathcal S = \{\text{zero}, \text{one}, \text{add}/\!/2, \text{mult}/\!/2\}$$

    \section{S-Struktur}
    Sei $\mathcal S$ eine Signatur. Eine $\mathcal S$-Struktur $\mathcal A$ besteht aus:
    \begin{itemize}
        \item Univserum $A$ mit $A \not= \emptyset$
        \item Für jedes Symbol eine Konstanten $c \in \mathcal{S}$ eine Interpretation $c^\mathcal A \in A$ von $c$.
        \item Für jedes Funktionssymbol $f/\!/n \in \mathcal S$ eine Interpretation $f^\mathcal A \colon A^n \to A$
        \item Für jedes Relationssymbol $R/n \in \mathcal S$ eine Interpretation $R^\mathcal A \subseteq A^n$
    \end{itemize}
    Hier ein Beispiel das ungefähr zu der Signatur oben passt:\\
    \begin{align*}
        A & = \{0, 1, 2, 3\} \\
        \text{zero}^\mathcal A & = 3\\
        \text{one}^\mathcal A & = 2\\
        \text{add}^\mathcal A(a,b) & = 0 && \text{for $a, b \in A$}\\
        \text{mult}^\mathcal A(a,b) & = a + b \text{ rest } 4 && \text{for $a, b \in A$}\\
        \text{Lt}^\mathcal N & = \{\langle a, a\rangle \colon a \in A\}
    \end{align*}


\end{document}