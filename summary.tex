\documentclass[twocolumn]{article}

\usepackage[margin=1.2cm]{geometry}
\usepackage{amssymb}
\usepackage{amsmath}
\usepackage{todonotes}
\usepackage{xcolor}
\usepackage{hyperref}
\usepackage{stmaryrd}
\graphicspath{ {./images/} }
\usepackage{hyperref}
\hypersetup{
    colorlinks,
    citecolor=black,
    filecolor=black,
    linkcolor=black,
    urlcolor=black
}
\renewcommand{\labelenumi}{(\roman{enumi})}

\title{Zusammenfassung: Logik für die Informatik }
\author{Rico Klimpel}
\date{\today}

\begin{document}
    \maketitle   
    \tableofcontents

    \setcounter{page}{1}

    \section*{Informationen}
    Zusammenfassung der Vorlesung Logik für die Informatik an der CAU Kiel aus dem Wintersemester 2019/2020, gehalten von Prof. Dr. Thomas Wilke. Ein Versuch die wichtigsten Aussagen ohne enorm lange Formalitäten drum herum knapp zu Papier zu bringen. Kein Anspruch auf Vollständigkeit. Geschrieben in \LaTeX.

    \clearpage

    \part{Aussagenlogik}
    

    \clearpage
    \part{Prädikatenlogik}

    \section{Syntax \& Semantik}

    \subsection{Signatur}
    \url{https://lili.informatik.uni-kiel.de/llocs/Signatures}\\\\
    Eine Signatur $\mathcal S$ besteht aus eine Menge $S$ von Symbolen und einer Funktion $\Sigma \colon S \to \mathbb N \cup \mathbb N \times \{1\}$.\\
    The Elemente von $S$ werden Symbole genannt und wie folgt eingeteilt:
    \begin{itemize}
        \item Ein Symbol $f$ mit $\Sigma(f) = \langle n, 1\rangle$ für $n > 0$ ist eine Funktionssymbol.\\
        Menge dieser Symbole: $\mathcal F_\Sigma$ oder einfach $\mathcal F$.
        \item Ein Symbol $R$ mit $\Sigma(R) = n$ für $n > 0$ ist ein Relationssymbol.\\
        Menge dieser Symbole: $\mathcal R_\Sigma$ oder $\mathcal R$.
        \item Ein Symbol $c$ mit $\Sigma(c) = \langle 0,1\rangle$ ist ein Symbol für eine Konstante.\\
        Menge dieser Symbole: $\mathcal C_\Sigma$ oder $\mathcal C$.
        \item Symbol $b$ mit $\Sigma(b) = 0$ ist ein Symbol für einen boolschen Wert. \\
        Menge dieser Symbole: $\mathcal B_\Sigma$ oder $\mathcal B$.
    \end{itemize}
    Im allgemeinen werden Signaturen mit $\mathcal B \neq \emptyset$ ignoriert (Signaturen ohne boolsche Werte). Keine Ahnung warum er das sagt.\\\\
    Beispiele:
    \begin{align*}
        S &= \{\text{zero}, \text{one}, \text{add}, \text{mult}\}\\
        \Sigma &= \{\text{zero} \mapsto \langle 0,1\rangle, \text{one} \mapsto \langle 0,1\rangle, \text{add} \mapsto \langle 2,1\rangle, \text{mult} \mapsto \langle 2,1\rangle\}
    \end{align*}
   Vereinfacht aufgeschrieben sieht das ganze so aus:
    $$\mathcal S = \{\text{zero}, \text{one}, \text{add}/\!/2, \text{mult}/\!/2\}$$

    \subsection{Struktur}
    \url{https://lili.informatik.uni-kiel.de/llocs/Structures}\\\\
    Sei $\mathcal S$ eine Signatur. Eine $\mathcal S$-Struktur $\mathcal A$ besteht aus:
    \begin{itemize}
        \item Univserum $A$ mit $A \not= \emptyset$
        \item Für jedes Symbol eine Konstanten $c \in \mathcal{S}$ eine Interpretation $c^\mathcal A \in A$ von $c$.
        \item Für jedes Funktionssymbol $f/\!/n \in \mathcal S$ eine Interpretation $f^\mathcal A \colon A^n \to A$
        \item Für jedes Relationssymbol $R/n \in \mathcal S$ eine Interpretation $R^\mathcal A \subseteq A^n$
    \end{itemize}
    Hier ein Beispiel das ungefähr zu der Signatur oben passt:\\
    \begin{align*}
        A & = \{0, 1, 2, 3\} \\
        \text{zero}^\mathcal A & = 3\\
        \text{one}^\mathcal A & = 2\\
        \text{add}^\mathcal A(a,b) & = 0 && \text{for $a, b \in A$}\\
        \text{mult}^\mathcal A(a,b) & = a + b \text{ rest } 4 && \text{for $a, b \in A$}\\
        \text{Lt}^\mathcal N & = \{\langle a, a\rangle \colon a \in A\}
    \end{align*}

    \subsection{Terme}
    \url{https://lili.informatik.uni-kiel.de/llocs/Syntax_of_first-order_logic#Formal_definition_of_terms}\\\\
    Induktive Defintion für alle Terme über eine Signatur $\mathcal{S}$, die auch $\mathcal{S}$-terms genannt wird:
    Basiselemente:\\
    \begin{itemize}
        \item Ein Baum mit nur einem Element das eine Variable der Prädikatenlogik enthält ist ein $\mathcal{S}$-term.
        \item Ein Baum mit nur einem Element das eine Konstante $c \in \mathcal{S}$ enthält ist ein $\mathcal{S}$-term.
    \end{itemize}
    Diese werden die atomaren $\mathcal{S}$-terme genannt.
    Induktionsregeln:\\
    \begin{itemize}
        \item Wenn  $f/\!/n \in \mathcal S$ eine Funktion und $t_0, \dots, t_{n-1}$ $\mathcal S$-terms sind, dann ist der Baum mit der Wurzel $f$ und den $n$ Teilbäumen $t_0$, ..., $t_{n-1}$ ein $\mathcal S$-term.
    \end{itemize}

    \subsection{Formeln}
    \url{https://lili.informatik.uni-kiel.de/llocs/Syntax_of_first-order_logic#Formal_definition_of_formulas}\\\\
    Induktive Defintion für alle Formeln über eine Signatur $\mathcal{S}$, die auch $\mathcal{S}$-formulas genannt wird:
    Basiselemente:\\
    \begin{itemize}
        \item Der einelementige Baum in dem das einzige Element eines der konstanten Symbole $\top$ oder $\bot$ ist, ist eine (prädikatenlogische) Formel.
        \item Wenn $t_0,t_1$ Terme sind, dann ist der Baum mit der Wurzel $\doteq$ und den Teilbäumen $t_0$ und $t_1$ eine Formel.
        \item Wenn $R/n \in \mathcal S$ eine Relation ist und $t_0,\dots,t_{n-1}$ Terme sind dann ist der Baum mit der Wurzel $R$ und den $n$ Teilbäumenm $t_0, \dots, t_{n-1}$ eine Formel.
    \end{itemize}
    Diese werden die atomaren Formeln genannt.
    Induktionsregeln:\\
    \begin{itemize}
        \item Wenn $C$ ein $n$-stelliger Junktor ist und $\varphi_0,\dots,\varphi_{n-1}$ Formeln sind, dann ist der Baum mit der Wurzel $C$ und den $n$ Teilbäumen $\varphi_0,\dots,\varphi_{n-1}$ eine Formel.
        \item Wenn $x_i$ eine Variable ist und $\varphi$ eine Formel, dann ist der Baum mit der Wurzel $\exists x_i$ oder der Wurzel $\forall x_i$ und dem Teilbaum $\varphi$ eine Formel.
    \end{itemize}

    \subsection{Interpretation von Termen}
    \url{https://lili.informatik.uni-kiel.de/llocs/Semantics_of_first-order_logic#Interpretation_of_terms}\\\\
    Sei $\mathcal S$ eine Signatur und $\mathcal A$ eine $\mathcal S$-Struktur. Für eine Belegung ($A$-Belegung) $\beta$, ist der Wert von jedem  $\mathcal S$-term $t$ in $\mathcal A$ unter $\beta$: $\llbracket t\rrbracket_\beta^\mathcal A$ defniert durch folgender Induktion.
    Basiselemente:
    \begin{itemize}
        \item Für alle $i \in \mathbb{N}$ gilt: $\llbracket x_i\rrbracket_\beta^\mathcal A = \beta(x_i)$.
        \item Für jedes $c \in C$ gilt: $\llbracket c\rrbracket_\beta^\mathcal A = c^{\mathcal A}$
    \end{itemize}
    Induktionsregel:
    \begin{itemize}
        \item Für alle $f/\!/n \in \mathcal F$ und die $\mathcal S$-terms $t_0, \dots, t_{n-1}$ gilt: $\llbracket f(t_0, \dots, t_{n-1})\rrbracket_\beta^\mathcal A = f^\mathcal A(\llbracket t_0\rrbracket_\beta^\mathcal A, \dots, \llbracket t_{n-1}\rrbracket_\beta^\mathcal A)$
    \end{itemize}

    \subsection{Interpretation von Formlen}
    \url{https://lili.informatik.uni-kiel.de/llocs/Semantics_of_first-order_logic#Interpretation_of_formulas}\\\\


    \subsection{Freie Variablen}
    \url{https://lili.informatik.uni-kiel.de/llocs/Free_variables}\\\\

    \subsection{Koinzidenzlemma}
    \url{https://lili.informatik.uni-kiel.de/llocs/Coincidence_lemma_(first-order_logic)}\\\\

    \section{Modelierung}

    \subsection{Relationen in Strukturen defnieren?}
    \url{https://lili.informatik.uni-kiel.de/llocs/Definable_relations}\\\\

    \subsection{Erfüllbarkeit einer Formel}
    \url{---}\\

    \section{Äquivalenz}

    \subsection{Äquivalenz von Formeln}
    \url{https://lili.informatik.uni-kiel.de/llocs/Formula_equivalence_(first-order_logic)}\\\\

    \subsection{Regeln der Prädikatenlogik}
    \url{---}\\

    \subsection{Quantorenregeln}
    \url{---}\\

    \subsection{Umbenennen von gebundenen Variablen}
    \url{---}\\

    \subsection{Scope von Quantoren}
    \url{https://lili.informatik.uni-kiel.de/llocs/Scope_of_quantifiers_in_first-order_logic}\\\\

    \subsection{Normalformen}

    \subsubsection{Boolsche Normalform}
    \url{https://lili.informatik.uni-kiel.de/llocs/Boolean_normal_form_(first-order_logic)}\\\\

    \subsubsection{Plenex Normalform}
    \url{https://lili.informatik.uni-kiel.de/llocs/Prenex_normal_form}\\\\

    \subsubsection{Konjunktive Normalform}
    \url{https://lili.informatik.uni-kiel.de/llocs/Conjunctive_normal_form_(first-order_logic)}\\\\

    \section{Folgerungsbeziehungen (Entailment)}

    \subsection{Folgerungsbeziehung}
    \url{---}\\

    \subsection{Beziehung zwischen Erfüllbarkeit und Folgerungsbeziehung}
    \url{---}\\

    \section{Beweissysteme}

    \subsection{Natürliches Beweissystem}
    \url{---}\\

    \subsubsection{Beweisregeln}
    \url{---}\\

    \subsubsection{Korrektheit \& Vollständigkeit}
    \url{---}\\

    \subsection{Resolutionsbeweise}
    \url{---}\\

    \subsubsection{Korrektheit \& Vollständigkeit}
    \url{---}\\

    \subsubsection{Verbindung zwischen Resolution und Logik-Programmierung}
    \url{---}\\

    \section{Kompaktheit}
    \url{---}\\






\end{document}